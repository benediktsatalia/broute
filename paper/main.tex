\documentclass[11pt,a4paper,notitlepage]{article}
\usepackage{amssymb}
\usepackage{amssymb}
\usepackage{geometry} 
\usepackage{float}
\usepackage{booktabs}
\usepackage{algorithm,algorithmic}
\usepackage{amsmath}
\usepackage{pdflscape}
\usepackage{url}
\usepackage{natbib}
\usepackage{graphicx,color}
\usepackage[utf8]{inputenc}
\usepackage{listings} % for inline code

\allowdisplaybreaks
\geometry{a4paper,left=1in,right=1in,top=1in,bottom=1in}

\newcommand{\ft}{\color{blue}}  % to show changes in revised version (F.T.)

\begin{document}

\title{Benchmark!!1}
\author{
Fabien Tricoire$^{1}$ \\[1ex]
 \small $^1$Institute for Transport and Logistics Management,
 Vienna University of Economics and Business\\
 \small Welthandelsplatz 1, 1020 Vienna, Austria\\
  \small \texttt{fabien.tricoire@wu.ac.at}\\[2ex]
}
\date{}
\maketitle

\begin{abstract}
  bla bla bla
\end{abstract}

\section{Introduction}
\label{sec:intro}

Some languages are easier, but how much performance do we lose by
using them?

Purposes: determine which language we can tell our students to
use to implement routing optimisation routines. Also determine which
language we can use ourselves.

Philosophy: not necessarily the best
implementation in the world, but a reasonably good one, i.e. one we
can expect from an average student. Similar algorithm in various languages.

What we measure: run time of the algorithm only. Everything else
(reading data, building initial solution) is not measured. Currently
Julia, Javascript, Pypy, Numba also measure compilation time.

In order to represent a tour in heuristics we restrict ourselves to
variable-size vectors, since they are necessary in most algorithms for
vehicle routing, for instance to insert or delete a vertex from a
tour.
% This is the reason why we do
% not provide a C implementation. It is of course possible to
% implement variable-size vectors using fixed-size arrays and
% bookkeeping; however this stands directly against our stated purposes
% and philosophy. Moreover, implementing a bug-free, efficient version
% of variable-size vectors is unlikely to result in something better
% than existing libraries.
Inserting and deleting elements in a vector has linear worst-case time
complexity. However, for heuristics it is still usually a good idea to
use vectors rather than, say, linked lists, since neighborhood
exploration typically requires to evaluate many moves then perform
only one of these moves. The complexity of performing a move is thus
secondary and unlikely to be a performance bottleneck. Additionally,
performing certain moves can actually be more efficient using a
vector than using a linked list, e.g. inverting a subsequence.

\subsection{Limitations}
Benchmarks are by definition limited in scope; any given benchmark
measurement is only valid within a certain environment. In our case
such environmental factors include the choice of compilers, compiler
versions, operating system and CPU family. Therefore all results are
to be taken as indicators but not absolute truths. The intent of this
study is not to declare a language the winner because it is
consistently faster than others by a few percents, but rather to
assess what kind of performance losses can be expected by, for
example, using Python instead of C++. OR to assess what kind of
performance gains can be achieved by using a JIT compiler for Python.

\section{The benchmarks}
We consider five different benchmark algorithms that perform tasks
commonly encountered in transport optimization. Together they capture
key aspects of computationally intensive tasks in heuristics and
exact methods for transport optimization. Additionally, we believe
that they also capture such aspects for other fields of application
of operations research, such as e.g. scheduling. For the sake of
simplicity, we consider the symmetric travelling salesperson problem
(TSP) as a base problem, i.e. input data are in the form of a
symmetric distance matrix while a solution is a permutation. However
some benchmark algorithms solve different problems than the TSP based on these
TSP data, as described below.

All five benchmarking algorithms described below are deterministic.
Additionally, the input provided to them is also deterministic,
i.e. each implementation receives the exact same inputs and performs
the same operations from these inputs. For that purpose, we generate
various instance files, each with a given size $n$, which represents
the number of vertices in the graph, and a number $p$ of
permutations. A permutation is a solution to the TSP. Each instance
file contains one $n \times n$ distance matrix as well as $p$ randomly
generated permutations, which are used as starting \emph{seeds} for
the algorithms (e.g. as starting solution for 2-opt). Since the same
instance files are given to 
each implementation, all implementations perform the same operations
and return the same result. In order to control result integrity we use a
mechanism similar to \emph{checksum}. The checksum calculation for
applying a certain algorithm using a given permutation as starting seed differs
based on the algorithm and is explained separately for each algorithm
below. The checksum calculation for an instance file is the sum of
checksum values over all permutations in that instance file.

Performances are measured per instance file, i.e. each time reported
is for running an algorithm $p$ times, using one distance matrix with the $p$
different seed permutations.

\subsection{2-opt}
One of the most commonly used heuristics in vehicle routing, 2-opt
improves a tour by performing 2-exchanges. A 2-exchange consists in
removing two edges from a tour and reconnecting the tour with two
other edges. It is equivalent to inverting a sub-sequence of the
tour, and can be performed in place using an array or vector solution
representation.

The checksum for a given permutation seed is calculated as the number of
improvements found while applying first-improvement 2-opt, using this
permutation as starting solution.

\subsection{Or-opt}


The checksum for a given permutation seed is calculated as the number of
improvements found while applying first-improvement Or-opt, using this
permutation as starting solution.

\subsection{Large neighbourhood search}

Note: we use a vector and not a set to store unplanned requests,
otherwise we take the risk of losing determinism.

The checksum for a given permutation seed is calculated as the total
cost of insertions performed while applying 10 iterations of LNS using
that permutation as starting solution.


\subsection{Dynamic programming for ESPPRC}
Motivation: column generation.

Mention here espprc-index and how rust does not allow multiple
references to other labels.

The checksum for a given permutation seed is the cost of the shortest
elementary path obtained when using that permutation seed to generate the
ESPPRC instance, truncated to its integer part.

\subsection{Maximum flow problem}
Motivation: branch-and-cut.

The checksum for a given permutation seed is the maximum flow value
obtained when using that permutation seed to generate the MFP
instance, truncated to its integer part. 

\section{Languages and implementations considered}
We consider six different programming languages: C++, Python, Java,
Julia, Rust and Javascript. Any rule of selection
is arbitrary in nature; however, we indicate below, for each of these
languages, the main reasons behind their selection. It is possible to
add other languages to this list in the future.

For each language, we may consider multiple \emph{implementations},
which differ by which mechanisms or library they use. For
instance with Python we can compare an implementation using native
lists against an implementation using Numpy arrays. Implementations are
detailed individually below for each language. The code for all
implementations can be found at
\url{https://github.com/fa-bien/routing-benchmark}.

\subsection{C++}
Two implementations: c++14 vs c++98.

Flat vs nested matrix.

static arrays too.

\subsection{Python}
Python vs Pypy vs Numpy vs Numba.
Flat vs nested matrix.

\subsection{Java}
Flat vs nested matrix.
static arrays too.

\subsection{Julia}
General performance tips: read the specific doc section! One unique
official document outlines how to write efficient code. there is also
an automated tool for that.

Flat vs nested matrix.

\subsection{Rust}

\subsection{JavaScript}
JavaScript is not expected to perform as well as, say, C++. However
there are several reasons why it can be appealing to implement routing
algorithms in JavaScript:
\begin{itemize}
\item Any program written in JavaScript can run on virtually any
  computer with a modern web browser, including smartphones.
\item Integrating JavaScript with a web interface is especially easy.
\item JavaScript engines have received considerable attention from
  major companies and are the subject of fierce competition on
  performance. As a result they have seen vast performance
  improvements over the years and the trend is likely to continue.
\end{itemize}

We use Node.js, which allows to run JavaScript programs from the
command line. This means that there is in fact not even the need for a
web browser. Node.js currently uses Google Chrome v8's JavaScript
engine~\cite{nodejs}.

Flat vs nested.

\subsection{Considerations on how to implement a distance matrix}
Perhaps the most straightforward way to implement a distance matrix in
a number of languages is to use nested arrays, i.e. each element of the
main array is an array representing a row of the distance matrix. This
is typically done in C++, Java, Python (although the structure is
officially called a \emph{list} and not an array), Javascript. C and
C++ use pointers to achieve that effect. All these languages use the
same C syntax for looking up values in the distance matrix: the
distance between vertices $i$ and $j$ using distance matrix $d$ is
written \lstinline{d[i][j]}.

Another easy way to implement a distance matrix is to use what we
call from here on a \emph{flat} distance matrix representation, which
is a single-dimensional array containing all distance values. Assuming
indices start at 0, the distance from $i$ to $j$ in array $d$ can be
coded as \lstinline{d[i*n+j]}, where $n$ is the total number of
vertices. This representation guarantees that the whole distance
matrix is stored in contiguous memory. Additionally, looking up in
arrays also takes time, so one lookup is better than two. The drawback
is that we have to pay a multiplication and an addition for each
lookup, and that lookups have to be wrapped in a function call for the
sake of readability. Such function calls can usually be inlined,
i.e. the function content is substituted to the function call, so
there should be no runtime penalty from using a function call. In
general we expect a speedup from using a flat representation. We
implemented some of the benchmarks with both flat and nested matrix
representation in order to determine if there are significant
performance differences.

There are a few cases where the above considerations on using a flat
representation do not apply: Python does not allow inlining, while
Numpy and Julia provide multi-dimensional arrays. Nonetheless we also
wrote a flat matrix version of these implementations, in order
to elicit any difference in performance. 

\subsection{Other general cross-language considerations}
Subjective considerations:
\begin{itemize}
\item Typically easier to program using Python, Julia or JavaScript.
\item Typically harder to program using Rust than anything else but
  might be due to lack of experience. However my experience in
  JavaScript is much lower than in C++ and I still did everything
  faster in JavaScript than in C++.
\end{itemize}


\section{Implementation-specific notes}
\subsection{Python}
Python does not allow to inline functions. Since using a flat matrix
requires to wrap matrix value lookup in a function, using a flat distance
matrix in Python is not beneficial. It would of course be possible to
inline by hand, i.e. formulate the correct 1-dimensional index in the
matrix for every lookup, but that would be impractical and defeat the
purposes and philosophy stated in Section~\ref{sec:intro}.
Running Python code that uses a flat representation with a JIT compiler
(e.g. Pypy, Numba) introduces automatic inlining and is likely to
remedy this.

On the ESPPRC benchmark, a label needs to store information about
which customers have already been visited in the partial path it
represents. In Python, preliminary testing reveals that using a set of
integers for that purpose is twice as fast as using a vector of
booleans. Similar preliminary testing in C++ reveals the opposite:
using an array of booleans is faster. This is the behaviour we would
expect in general, the observation on Python are the surprising ones
here. There are two reasons for this expectation: (i) copying an array,
represented in contiguous memory, can in general be performed faster
than copying a set structure and (ii) lookup in an array is performed
in $\mathcal{O}(1)$ while lookup in a set is performed in
$\mathcal{O}(log(n))$.

\subsection{Julia}
Julia has native multi-dimensional arrays, which may be more efficient
than a flat representation. We will determine which version to use
based on experiment. Since Julia's arrays start with index 1, the code
for flat representation lookup of the distance between vertices $i$
and $j$ in matrix $d$ is \lstinline{d[i*n-n+j]}. It is worth noting
that it involves one subtraction on top of the addition and
multiplication used for 0-indexed languages.


Using two threads instead of 1 improves performance by about 10\%,
although it is unclear what the second thread is used for.


\subsection{JavaScript}
All benchmarks are repetitive in nature, as they successively perform similar
operations with different input data. In JavaScript, when our
implementation loads 40 instances and performs the same
benchmarking operations on all 40 them, we observe a significant performance
hit after a few instances (2-5 times slower). In order to remedy this,
we re-start the program individually for each instance. Advanced
knowledge of this specific JavaScript engine might allow to remove that
performance hit, but this is clearly outside the scope of this
study. However it is worth noting that this can be a concern in general.

\subsection{Numba}
Numba is not yet feature-complete for Numpy. A direct consequence is that some
benchmarks cannot be implemented for Numba. For example
\lstinline{numpy.insert()} is not implemented, but this function is
needed for the $LN$ benchmark. Therefore the $LNS$ benchmark cannot be
implemented with Numba. Additionally, we were not able to implement
the ESPPRC benchmark in a satisfying manner. In our experience, Numba
error messages are sometimes unrelated to the issue causing them, or
too obscure to make sense, keeping in mind the premises of this
paper. It would certainly be possible to implement a dynamic
programming algorithm for the ESPPRC benchmark using Numba's JIT
compilationfeatures, however our conclusion is that it would not be
possible to do so without expert knowledge of Numba.

Still, Numba is significant enough to be
benchmarked and its current performances are an indication of what to
expect in years to come, as the project matures.

\subsection{Java}
Mention inconsistencies between Java versions: code written for Java 8
does not compile with Java 11. Because Pair is not in the API any
more.

\subsection{Rust}
Issues:
\begin{itemize}
\item Changing from i32 to i64 is problematic somehow
\item Not possible to have refs to both successors and predecessor in
  Label, hence had to handle indices by hand in specific espprc implementation
\end{itemize}

\section{Experiments}
All programs are run concurrently on an AMD Ryzen 5 3600 6-Core
Processor at 3.6 GHz with 16 GB RAM running Linux. Twelve threads can
run in parallel, however in our setting at most 5 threads are used at
any given time.

Mention here compilers and compiler versions used.

\subsection{Impact of using a flat matrix}
\subsection{Impact of using static arrays}
\subsection{A comparison of Python implementations}
\subsection{A comparison of C++ implementations}
Also add a clang++ vs g++ comparison?
\subsection{General cross-language comparison}
\subsection{Cross-language comparison: fast languages}

\section{Discussion}
The question of what programming language to use when implementing
optimization algorithms is one that, by essence, does not have a final
answer.

\section*{Acknowledgements}
The author is grateful to Sebastian Leitner for his valuable support
in relation with the Rust implementation.

\bibliographystyle{plain}
\bibliography{benchmark}

\end{document}